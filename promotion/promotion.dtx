% \iffalse meta-comment
%
% Copyright (C) 2020 by Joel Coffman
%
% This file may be distributed and/or modified under the
% conditions of the LaTeX Project Public License, either version 1.2
% of this license or (at your option) any later version.
% The latest version of this license is in:
%
%   http://www.latex-project.org/lppl.txt
%
% and version 1.2 or later is part of all distributions of LaTeX
% version 1999/12/01 or later.
%
% \fi
%
% \iffalse
%<*driver>
\documentclass{ltxdoc}

\usepackage{enumitem}
\usepackage{glossaries}
\usepackage{hyperref}
\usepackage{microtype}
\usepackage{minted}

\usepackage{minted-doc}

\usepackage{promotion}


% enumitem
\setlist{
  noitemsep,
}

% glossaries
\loadglsentries{../acronyms}


\input{.version}


\EnableCrossrefs
\CodelineIndex
\RecordChanges

\begin{document}
  \DocInput{promotion.dtx}
\end{document}
%</driver>
% \fi
%
% \CheckSum{0}
%
% \CharacterTable
% {Upper-case   \A\B\C\D\E\F\G\H\I\J\K\L\M\N\O\P\Q\R\S\T\U\V\W\X\Y\Z
% Lower-case    \a\b\c\d\e\f\g\h\i\j\k\l\m\n\o\p\q\r\s\t\u\v\w\x\y\z
% Digits        \0\1\2\3\4\5\6\7\8\9
% Exclamation   \!     Double quote  \"     Hash (number) \#
% Dollar        \$     Percent       \%     Ampersand     \&
% Acute accent  \'     Left paren    \(     Right paren   \)
% Asterisk      \*     Plus          \+     Comma         \,
% Minus         \-     Point         \.     Solidus       \/
% Colon         \:     Semicolon     \;     Less than     \<
% Equals        \=     Greater than  \>     Question mark \?
% Commercial at \@     Left bracket  \[     Backslash     \\
% Right bracket \]     Circumflex    \^     Underscore    \_
% Grave accent  \`     Left brace    \{     Vertical bar  \|
% Right brace   \}     Tilde         \~}
%
%
% \changes{0.1.0}{2020/10/09}{Initial version}
%
% \GetFileInfo{promotion.sty}
%
% \DoNotIndex{\#,\$,\%,\&,\@,\\,\{,\},\^,\_,\~,\ }
% \DoNotIndex{\@ne}
% \DoNotIndex{\advance,\begingroup,\catcode,\closein}
% \DoNotIndex{\closeout,\day,\def,\edef,\else,\empty,\endgroup}
% \DoNotIndex{\global,\let,\relax}
%
% \title{The \textsf{promotion} package\thanks{This
% document corresponds to \textsf{promotion~\fileversion-\version, dated \filedate.}}}
% \author{Joel Coffman\\\texttt{joel.coffman@usafa.edu}}
%
% \maketitle
%
% \begin{abstract}
% This package supports applications for academic promotion at \gls{USAFA}.
% The package itself is effectively a ``metapackage'' to avoid the need to know which package actually provides the various commands used to typeset the application.
% \end{abstract}
%
% \section{Usage}\label{section:usage}
%
% Load this package in the document preamble:
% \begin{VerbatimOut}[gobble=2]{minted/use-promotion-package.out}
%   \usepackage{promotion}
% \end{VerbatimOut}
% \inputminted{latex}{minted/use-promotion-package.out}
%
% \subsection{Macros}\label{section:macros}
%
% \DescribeMacro{rank}
% This package provides the |rank| macro to define the applicant's (prospective) academic rank.
%
% \medskip
% \noindent
% |\rank|\marg{academic rank}\par
% \medskip
%
% \noindent
% This macro should only be used in the document preamble.
%
% \DescribeMacro{semester}
% The |semester| macro is handy for listing the semester in which a course has been taught:
%
% \medskip
% \noindent
% |\semester|\marg{season and year}\par
% \medskip
%
% \noindent
% The semester is typeset flush right on the same line if there is sufficient space and otherwise on the following line.
% For example,
% \begin{VerbatimOut}[gobble=2]{minted/semester.out}
%   \begin{description}
%     \item[Leadership 100]
%         Foundations for Personal Leadership
%         \semester{Fall 2020}
%     \item[Learn Strat 101]
%         Learning Strategies for Academic and Career Success
%         \semester{Spring 2021}
%   \end{description}
% \end{VerbatimOut}
% \inputminted{latex}{minted/semester.out}
% is typeset as follows:
% \input{minted/semester.out}
% where the longer course title forces the semester information to appear on the following line.
%
% \StopEventually{
%   \bibliographystyle{plainurl}
%   \bibliography{references}
%
%   \PrintChanges
%   \PrintIndex
% }
%
% \appendix
%
% \iffalse
%<*package>
% \fi
%
% \section{Implementation}
% This section documents the implementation of the package.
%
% Require \LaTeXe.
%    \begin{macrocode}
\NeedsTeXFormat{LaTeX2e}
%    \end{macrocode}
% Identify package and version.
%    \begin{macrocode}
\ProvidesPackage{promotion}[%
    2020/10/09 %
    v0.1.0 %
    Package for academic promotion material%
]
%    \end{macrocode}
%
% \subsection{Packages}
% Load packages required by this one.
%    \begin{macrocode}
\RequirePackage{record}
\RequirePackage{statement}
\RequirePackage{supplements}
%    \end{macrocode}
%
% \subsection{Macros}
% This section describes the macros defined by this package.
%
% Define the |rank| macro, which stores the (prospective) academic rank in |\@rank| similar to how the title, author, and date are handled by document classes.
%    \begin{macrocode}
\providecommand*{\rank}[1]{%
  \gdef\@rank{#1}%
}
%    \end{macrocode}
%
% Define the |semester| macro.
%    \begin{macrocode}
\newcommand*{\semester}[1]{%
%    \end{macrocode}
% The implementation is courtesy of \emph{The TeXbook}~\cite[Chapter~14]{knuth1984texbook} (search for ``Bourbaki'') and typesets the semester flush right.
% If there isn't sufficient space on the current line, then it appears on the following line.
%    \begin{macrocode}
  \unskip\nobreak\hfil\penalty50
  \hskip1em\hbox{}\nobreak\hfil#1
  \parfillskip=0pt \finalhyphendemerits=0 \par%
%    \end{macrocode}
%    \begin{macrocode}
}
%    \end{macrocode}
%
% \iffalse
%</package>
% \fi
%
% \Finale
\endinput

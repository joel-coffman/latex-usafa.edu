% \iffalse meta-comment
%
% Copyright (C) 2020 by Joel Coffman
%
% This file may be distributed and/or modified under the
% conditions of the LaTeX Project Public License, either version 1.2
% of this license or (at your option) any later version.
% The latest version of this license is in:
%
%   http://www.latex-project.org/lppl.txt
%
% and version 1.2 or later is part of all distributions of LaTeX
% version 1999/12/01 or later.
%
% \fi
%
% \iffalse
%<*driver>
\documentclass{ltxdoc}

\usepackage{bibentry}
\usepackage{minted}
\usepackage{microtype}

\usepackage{hyperfix}

\usepackage{record}


% bibentry
\robustify{\bibentry}

\usepackage{microtype}

\usepackage{hyperfix}

\usepackage{record}


% bibentry
\robustify{\bibentry}

\AtBeginDocument{
  \bibliographystyle{plainurl}
  \nobibliography{references}
}

% minted
\setminted{
  autogobble,
  breaklines,
}


% macros / environments
\providecommand*{\meta}[1]{%
  \textlangle\,\emph{#1}\,\textrangle%
}


\input{.version}


\EnableCrossrefs
\CodelineIndex
\RecordChanges

\begin{document}
  \DocInput{record.dtx}
\end{document}
%</driver>
% \fi
%
% \CheckSum{0}
%
% \CharacterTable
% {Upper-case   \A\B\C\D\E\F\G\H\I\J\K\L\M\N\O\P\Q\R\S\T\U\V\W\X\Y\Z
% Lower-case    \a\b\c\d\e\f\g\h\i\j\k\l\m\n\o\p\q\r\s\t\u\v\w\x\y\z
% Digits        \0\1\2\3\4\5\6\7\8\9
% Exclamation   \!     Double quote  \"     Hash (number) \#
% Dollar        \$     Percent       \%     Ampersand     \&
% Acute accent  \'     Left paren    \(     Right paren   \)
% Asterisk      \*     Plus          \+     Comma         \,
% Minus         \-     Point         \.     Solidus       \/
% Colon         \:     Semicolon     \;     Less than     \<
% Equals        \=     Greater than  \>     Question mark \?
% Commercial at \@     Left bracket  \[     Backslash     \\
% Right bracket \]     Circumflex    \^     Underscore    \_
% Grave accent  \`     Left brace    \{     Vertical bar  \|
% Right brace   \}     Tilde         \~}
%
%
% \changes{0.0.1}{2020/10/09}{Initial version}
%
% \GetFileInfo{record.sty}
%
% \DoNotIndex{\#,\$,\%,\&,\@,\\,\{,\},\^,\_,\~,\ }
% \DoNotIndex{\@ne}
% \DoNotIndex{\advance,\begingroup,\catcode,\closein}
% \DoNotIndex{\closeout,\day,\def,\edef,\else,\empty,\endgroup}
% \DoNotIndex{\global,\let,\relax}
%
% \title{The \textsf{record} package\thanks{This
% document corresponds to \textsf{record~\fileversion-\version, dated \filedate.}}}
% \author{Joel Coffman\\\texttt{joel.coffman@usafa.edu}}
%
% \maketitle
%
% \begin{abstract}
% Package for scholarship and service records
% \end{abstract}
%
% \section{Usage}\label{section:usage}
%
% \DescribeEnv{publications}
% The |publications| environment\dots
% For example,
% \begin{VerbatimOut}[gobble=2]{minted/publications.out}
% \begin{publications}[
%     applicant=\meta{Applicant},
%     rank=\meta{Academic Rank},
% ]
%   \publication[
%     institution=USAFA,
%     peer review,
%     teaching,
%     type=book,
%   ]{jones2014building}{
%     \begin{itemize}
%       \item
%       This work is an outgrowth of the Course Design Retreats that I co-created and co-led during the sumemrs of 2008--2014.
%
%       \item
%       I wrote 3 of the 10 chapters and worked closely with my co-authors on the entire publication.
%       This involved editing, revising, referencing, and illustrating the book.
%       The author team spent three years working closely together to bring this book from conception to bound book.
%     \end{itemize}
%   }{
%     \begin{itemize}
%       \item
%       Our innovative book breaks new ground in the area of design college courses.
%       Its novel workbox approach is used in faculty development workshops thorughout the country.
%
%       \item
%       Stylus Publishing is one of the top two educational publisters in the US and the UK (see URL or attached supplemental material for their extensive catalog).
%     \end{itemize}
%   }
% \end{publications}
% \end{VerbatimOut}
% \inputminted{latex}{minted/publications.out}
% produces\\
% \input{minted/publications.out}\par
%
% \DescribeEnv{presentations/workshops}
% The |presentations/workshops| environment\dots
% \begin{VerbatimOut}[gobble=2]{minted/presentations--workshops.out}
% \begin{presentations/workshops}[
%     applicant=\meta{Applicant},
%     rank=\meta{Academic Rank},
% ]
%   \presentation[
%     type=Faculty Development Workshop,
%     teaching,
%     service,
%   ]{
%     \begin{itemize}
%       \item 2 June 2017
%       \item Co-presented with Dr.~Steve Jones (USAFA)
%       \item \emph{Building a Pathway for Student Learning: How Flawed Course Design Leads to Learning Failures}
%       \item Advanced Space Operations Schools (AFASOS), Peterson AFB, Colorado Springs, CO
%     \end{itemize}
%   }{
%     \begin{itemize}
%       \item A day-long workshop (0800--1630)
%       \item We presented this workshop to 25 instructors in the Advanced Space Operations School.
%       \item We supported the education and training mission of this AF schoolhouse. Our strategy tapped into their experiences with learning failures in their courses and then connected them to the six elements of our course design model. The afternoon was spent looking for solutions to these learning breakdowns and failures.
%       \item I co-created and co-presented this workshop with Dr.~Steve Jones.
%     \end{itemize}
%   }{
%     \begin{itemize}
%       \item Steve and I were invited to provide expert support to the education and training mission of AFASOS.
%       \item While AFASOS has a robust instructor training component, like most college instructors, they receive little training on how to design effective courses.
%       \item In this way we helped fill a gap in their training and education. Helping instructors solve their teaching challenges is always rewarding and improves the training of airmen entering their programs.
%     \end{itemize}
%   }
% \end{presentations/workshops}
% \end{VerbatimOut}
% \inputminted{latex}{minted/presentations--workshops.out}
% produces\\
% \input{minted/presentations--workshops.out}\par
%
% \DescribeEnv{service}
% The |service| environment\dots
% \begin{VerbatimOut}[gobble=2]{minted/service.out}
% \begin{service}[
%     applicant=\meta{Applicant},
%     rank=\meta{Academic Rank},
% ]
%   \activity[
%     date=2012--present,
%     organization={Member, Editorial Board Peer Reviewer, College Teaching Journal},
%     role=Review manuscripts submitted for publication,
%     type={Disciplinary service to the higher education community in the areas of teaching and learning},
%     sphere=National,
%     frequency={During the past 5 years, I have reviewed 24 manuscripts},
%     length={Each manuscript took $\approx 6$ hours to read, anlyze, synthesize, and write},
%     teaching,
%   ]{
%     I regularly receive an invitation to review manuscripts in my areas of expertise for one of the premier teaching journals.
%     My reviews require me to read, evaluate, and write a response to accept or reject the manuscript.
%   }{
%     I serve as an important gatekeeper for publications.
%     I have rejected 95\% of the manuscripts I have received.
%     This required me to write an extensive response to justify my rejection, as well as provide feedback and constructive criticism to the authorsl
%     My recommendations have been overridden by the Editor-in-Chief only one time.
%   }
%   \activity[
%     date=2016--present,
%     organization={USAFA Center for Educational Innovation (CEI)},
%     role={Learning Community Mentor, Faculty Orientation},
%     type=Institutional,
%     sphere=USAFA,
%     frequency={One week during July orientation; met with learning community (8 faculty) about every two weeks throughout the academic year},
%     length={Each meeting took $\approx 2$ hours to prepare and facilitate},
%   ]{
%     I serve as a teaching mentor for new and returning faculty members durig faculty orientation.
%     In this role I prompt and facilitate discussion, elicit suggestions, and give advice to incoming faculty.
%     During the academic year I provide discussion prompts and help faculty gain the necessary teaching skills and resolve teaching-learning challenges.
%   }{
%     I serve an essential role in accelerating the development of new and returning instructors that will, in turn, potentially impact the development of $\approx 200$ cadets (8 faculty $\times$ 24 cadets) here at USAFA.
%   }
% \end{service}
% \end{VerbatimOut}
% \inputminted{latex}{minted/service.out}
% produces\\
% \documentclass[11pt,crop=false]{standalone}

\usepackage{record}

\usepackage{application}


\begin{document}
\smaller

\begin{service}
  \activity[
    date=2012--present,
    organization={Member, Editorial Board Peer Reviewer, College Teaching Journal},
    role=Review manuscripts submitted for publication,
    type={Disciplinary service to the higher education community in the areas of teaching and learning},
    sphere=National,
    frequency={During the past 5 years, I have reviewed 24 manuscripts},
    length={Each manuscript took $\approx 6$ hours to read, anlyze, synthesize, and write},
    teaching,
  ]{
    I regularly receive an invitation to review manuscripts in my areas of expertise for one of the premier teaching journals.
    My reviews require me to read, evaluate, and write a response to accept or reject the manuscript.
  }{
    I serve as an important gatekeeper for publications.
    I have rejected 95\% of the manuscripts I have received.
    This required me to write an extensive response to justify my rejection, as well as provide feedback and constructive criticism to the authorsl
    My recommendations have been overridden by the Editor-in-Chief only one time.
  }
  \activity[
    date=2016--present,
    organization={USAFA Center for Educational Innovation (CEI)},
    role={Learning Community Mentor, Faculty Orientation},
    type=Institutional,
    sphere=USAFA,
    frequency={One week during July orientation; met with learning community (8 faculty) about every two weeks throughout the academic year},
    length={Each meeting took $\approx 2$ hours to prepare and facilitate},
  ]{
    I serve as a teaching mentor for new and returning faculty members during faculty orientation.
    In this role I prompt and facilitate discussion, elicit suggestions, and give advice to incoming faculty.
    During the academic year I provide discussion prompts and help faculty gain the necessary teaching skills and resolve teaching-learning challenges.
  }{
    I serve an essential role in accelerating the development of new and returning instructors that will, in turn, potentially impact the development of $\approx 200$ cadets (8 faculty $\times$ 24 cadets) here at USAFA.
  }
\end{service}

\end{document}
\par
%
% \StopEventually{
%   \PrintChanges
%   \PrintIndex
% }
%
% \appendix
%
% \iffalse
%<*package>
% \fi
%
% \section{Implementation}
% This section documents the implementation of the package.
%
% Require \LaTeXe.
%    \begin{macrocode}
\NeedsTeXFormat{LaTeX2e}
%    \end{macrocode}
% Identify package and version.
%    \begin{macrocode}
\ProvidesPackage{record}[%
    2020/10/09 %
    v0.0.1 %
    Package for scholarship and service records%
]
%    \end{macrocode}
%
% \subsection{Packages}
% Load packages required by this one.
%    \begin{macrocode}
\RequirePackage{amssymb}
\RequirePackage{booktabs}
\RequirePackage{enumitem}[2018-11-30]
\RequirePackage{environ}
\RequirePackage[pass]{geometry}
\RequirePackage{ltablex}
\RequirePackage{pdflscape}
\RequirePackage{pgfkeys}
\RequirePackage{pgfopts}
\RequirePackage{relsize}
\RequirePackage{stackengine}
\RequirePackage{xpatch}
%    \end{macrocode}
%
% \subsection{Macros}
% This section describes the macros in the \textsf{record} package.
%    \begin{macrocode}
\pgfkeys{% definitions
  record/.cd,
  applicant/.store in=\record@applicant,
  applicant/.value required,
  rank/.store in=\record@rank,
  rank/.value required,
}
\ProcessPgfOptions*

\xapptocmd{\author}{%
  \@ifundefined{record@applicant}{%
    \pdfstringdef{\record@applicant}{#1}%
  }{}%
}{}{}


\xpatchcmd{\TX@endtabularx}{\end{tabularx}}{\endtabularx\endgroup}
    {}{}  % https://tex.stackexchange.com/q/196993/#197000

\let\record@itemize@old=\itemize\relax
\let\record@enditemize@old=\enditemize\relax
\newenvironment{record@itemize}[1][]{%
  \begin{minipage}[t]{\linewidth}%
    \raggedright%
    \record@itemize@old[
        after=\strut,
        left=0pt,
        nosep,
        #1,
    ]
}{%
    \record@enditemize@old%
  \end{minipage}
}


\newcommand*{\record@getvalue}[1]{%
  \ifcsname record@#1\endcsname%
    \csname record@#1\endcsname%
  \else%
    \errmessage{Missing required argument `#1' for record}%
  \fi%
}%

\NewEnviron{record@record}[4][]{%
  \let\itemize=\record@itemize\relax
  \let\enditemize=\endrecord@itemize\relax
  \newgeometry{
    margin=1in,
  }%
  \begin{landscape}%
  \begin{tabularx}{\linewidth}{#3}
    % header
    #4\\
    \if\relax\detokenize{#2}\relax\else\addlinespace#2\\\fi
    \midrule
    \endfirsthead
    #4\\
    \midrule
    \endhead
    % footer
    \bottomrule
    \endfoot
    \BODY
  \end{tabularx}%
  \end{landscape}%
  \restoregeometry%
}

\pgfkeys{% default
  record/publications/.cd,
  .search also={/record},
}

\newenvironment{publications}[1][]{%
  \pgfkeys{record/publications/.cd,#1}%
  \record@record{
    & Date, all contributing authors, title, publication, publisher name, location, pages, and URL if accessible online
    & Where did you do this work?\\[\baselineskip] Peer Reviewed?
    & Scholarly book, textbook book chapter, research paper, review paper, article, blog, etc.
    & \begin{itemize}
        \item \emph{Describe your scholarly or intellectual contribution to the publication} such as senior author, editor, reviewer, etc.
        \item \emph{What specific work did you do?} (e.g., designed research, performed research, contributed new analytic tools, analyzed data, wrote the paper / book / review)
      \end{itemize}
    & \begin{itemize}
        \item \emph{How important is this publication in your field?} What value did it add to the field? If available, include the citation count for your pub, or number of books sold.
        \item \emph{How prestigious or selective is the publication outlet?} If available, provide concrete evidence for this such as impact factor of the journal, acceptance rate, other quality indicators, and overall reputation within the field.
      \end{itemize}
    & \Longunderstack{T e a c h i n g}
    & \Longunderstack{S e r v i c e}
  }{%
      r
      >{\raggedright}p{1.5in}
      >{\arraybackslash\centering}p{0.75in}
      >{\raggedright}p{0.75in}
      >{\raggedright}p{2in}
      >{\raggedright}X
      c
      c
  }{%
    \multicolumn{8}{c}{\larger[2]\record@getvalue{applicant} -- Applying for \record@getvalue{rank}\par}\\\addlinespace
    \multicolumn{8}{c}{\bfseries\larger[3] Scholarship Record: Publications\par}\\\addlinespace
    \multicolumn{8}{c}{\larger (since last promotion)\par}\\\addlinespace
    \toprule
      {\bfseries \#}
    & {\bfseries\parbox[b]{\linewidth}{\centering Citation}}
    & {\bfseries\parbox[b]{\linewidth}{\centering Institution / Peer Review}}
    & {\bfseries\parbox[b]{\linewidth}{\centering Type}}
    & {\bfseries\parbox[b]{\linewidth}{\centering Contribution}}
    & {\bfseries\parbox[b]{\linewidth}{\centering Significance and Impact}}
    & {\bfseries T}
    & {\bfseries S}
  }
}{%
  \endrecord@record
}


\newif\ifrecord@entry@peerreview%
\newif\ifrecord@entry@service%
\newif\ifrecord@entry@teaching%
\pgfkeys{% definitions
  record/entry/.cd,
  peer review/.is if=record@entry@peerreview,
  service/.is if=record@entry@service,
  teaching/.is if=record@entry@teaching,
  type/.store in=\record@entry@type,
  type/.value required,
}

\newcommand*{\record@entry@getvalue}[2][entry]{
  \ifcsname record@#1@#2\endcsname%
    \csname record@#1@#2\endcsname%
  \else%
    \ifcsname record@entry@#2\endcsname%
      \csname record@entry@#2\endcsname%
    \else%
      \errmessage{Missing required argument `#2' for #1}%
    \fi%
  \fi%
}%

\newcounter{record@publication}

\pgfkeys{% definitions
  record/publication/.cd,
  .search also={/record/entry},
  institution/.store in=\record@publication@institution,
  institution/.value required,
  number/.default={%
    \stepcounter{record@publication}%
    \arabic{record@publication}%
  },
  number/.store in=\record@publication@number,
  % defaults
  number,
}
\newcommand{\publication}[4][]{%
    \pgfkeys{record/publication/.cd,#1}% arguments
    \record@entry@getvalue[publication]{number}
  & \bibentry{#2}
  & \pgfkeys{record/publication/.cd,#1}% arguments
    \record@publication@institution\\[\baselineskip]
    \ifrecord@entry@peerreview Yes\else No\fi
  & \pgfkeys{record/publication/.cd,#1}% arguments
    \record@entry@type
  & \parskip=1ex #3
  & \parskip=1ex #4
  & \pgfkeys{record/publication/.cd,#1}% arguments
    \ifrecord@entry@teaching\checkmark\fi
  & \pgfkeys{record/publication/.cd,#1}% arguments
    \ifrecord@entry@service\checkmark\fi
  \\
}


\pgfkeys{% default
  record/presentations--workshops/.cd,
  .search also={/record},
}

\newenvironment{presentations/workshops}[1][]{%
  \pgfkeys{record/presentations--workshops/.cd,#1}%
  \record@record[#1]{
    & \begin{itemize}
        \item Date
        \item Presenters
        \item Title
        \item Conference or venue, location
      \end{itemize}
    & conference oral presentation, faculty development workshop, poster at a conference, TED talk, etc.
    & Describe your  workshop or presentation to the academic community:
      \begin{itemize}
        \item How long was it? (e.g., one hour, 4-hr, day, 2 days)
        \item Who was the audience? How many attended?
        \item Why were you there? What was your purpose? Strategy?
        \item What specific work did you do? (e.g., designed the workshop, presented the workshop)
      \end{itemize}
    & \begin{itemize}
        \item Was this an invited presentation or keynote?
        \item How prestigious and/or selective is the conference/workshop? (i.e., what is the acceptance rate for conference submissions?)
        \item What was the significance and/or impact of your presentation to the audience?
      \end{itemize}
    & \Longunderstack{T e a c h i n g}
    & \Longunderstack{S e r v i c e}
  }{%
      r  % #
      >{\raggedright}p{1.25in}  % Presentation / Workshop
      >{\raggedright}p{1.25in}  % Type
      >{\raggedright}p{2.5in}  % Contribution
      >{\raggedright}p{2.5in}  % Significance and Impact
      c  % Teaching
      c  % Service
  }{%
    \multicolumn{7}{c}{\larger[2]\record@getvalue{applicant} -- Applying for \record@getvalue{rank}\par}\\\addlinespace
    \multicolumn{7}{c}{\bfseries\larger[3] Scholarship Record: Presentations / Workshops\par}\\\addlinespace
    \multicolumn{7}{c}{\larger (since last promotion)\par}\\\addlinespace
    \toprule
      {\bfseries \#}
    & {\bfseries\parbox[b]{\linewidth}{\centering Presentation / Workshop}}
    & {\bfseries\parbox[b]{\linewidth}{\centering Type}}
    & {\bfseries\parbox[b]{\linewidth}{\centering Contribution}}
    & {\bfseries\parbox[b]{\linewidth}{\centering Significance and Impact}}
    & {\bfseries T}
    & {\bfseries S}
  }%
}{%
  \endrecord@record%
}


\newcounter{record@presentation}

\pgfkeys{% definitions
  record/presentation/.cd,
  .search also={/record/entry},
  number/.default={%
    \stepcounter{record@presentation}%
    \arabic{record@presentation}%
  },
  number/.store in=\record@presentation@number,
  % defaults
  number,
}
\newcommand{\presentation}[4][]{%
    \pgfkeys{record/presentation/.cd,#1}% arguments
    \record@entry@getvalue[presentation]{number}
  & \parskip=1ex #2
  & \pgfkeys{record/presentation/.cd,#1}% arguments
    \record@entry@type
  & \parskip=1ex #3
  & \parskip=1ex #4
  & \pgfkeys{record/presentation/.cd,#1}% arguments
    \ifrecord@entry@teaching\checkmark\fi
  & \pgfkeys{record/publication/.cd,#1}% arguments
    \ifrecord@entry@service\checkmark\fi
  \\
}
\let\workshop=\presentation\relax


\pgfkeys{% default
  record/service/.cd,
  .search also={/record},
}

\newenvironment{service}[1][]{%
  \pgfkeys{record/service/.cd,#1}%
  \record@record{
    & \begin{itemize}
        \item Date(s) of service
        \item The committee or organization your served
        \item Your role such as organizer, chair, or member
      \end{itemize}
    & \begin{itemize}
        \item Type -- institutional or disciplinary
        \item Sphere -- USAFA, regional, national, or international
      \end{itemize}

      \emph{Note: Do not include department-level additional duties or part of your job.}
    & \begin{itemize}
        \item Frequency -- one time event or recurring; annual, biannual, 1 year, etc.
        \item Length -- $\approx$ number of hours to prepare / perform
      \end{itemize}
    & \begin{itemize}
        \item Describe your service contribution
        \item What specific work did you do?
      \end{itemize}

      \emph{Note: You can leave this and the significance column blank if the service activities are common knowledge such as AAOCA or Executive Officer}
    & \begin{itemize}
        \item How important was this service?
        \item In what ways did you specifically impact the success of the committee's or organization's mission?
        \item Who was positively impacted?
        \item How many were impacted or potentially impacted?
      \end{itemize}
    & \Longunderstack{T e a c h i n g\strut}
    & \Longunderstack{S c h o l a r s h i p\strut}
  }{
    r  % #
    >{\raggedright\arraybackslash}p{1.5in}  % Service
    >{\raggedright\arraybackslash}p{1.5in}  % Type & Sphere
    >{\raggedright\arraybackslash}p{1in}  % Frequency & Length
    >{\raggedright\arraybackslash}X  % Contribution
    >{\raggedright\arraybackslash}X  % Significance and Impact
    c  % Teaching
    c  % Service
  }{
    \multicolumn{8}{c}{\larger[2]\record@getvalue{applicant} -- Applying for \record@getvalue{rank}\par}\\\addlinespace
    \multicolumn{8}{c}{\bfseries\larger[3] Academic Service Record\par}\\\addlinespace
    \multicolumn{8}{c}{\larger (since last promotion)\par}\\\addlinespace
    \toprule
      {\bfseries \#}
    & {\bfseries\parbox[b]{\linewidth}{\centering Service}}
    & {\bfseries\parbox[b]{\linewidth}{\centering Type \& Sphere}}
    & {\bfseries\parbox[b]{\linewidth}{\centering Frequency\\\& Length}}
    & {\bfseries\parbox[b]{\linewidth}{\centering Contribution}}
    & {\bfseries\parbox[b]{\linewidth}{\centering Significance and Impact}}
    & {\bfseries T}
    & {\bfseries S}
  }%
}{%
  \endrecord@record%
}


\newcounter{record@activity}

\newif\ifrecord@activity@scholarship
\pgfkeys{% definitions
  record/activity/.cd,
  .search also={/record/entry},
  date/.store in=\record@activity@date,
  date/.value required,
  frequency/.store in=\record@activity@frequency,
  frequency/.value required,
  institution/.store in=\record@activity@institution,
  institution/.value required,
  length/.store in=\record@activity@length,
  length/.value required,
  number/.default={%
    \stepcounter{record@activity}%
    \arabic{record@activity}%
  },
  number/.store in=\record@activity@number,
  organization/.store in=\record@activity@organization,
  organization/.value required,
  role/.store in=\record@activity@role,
  role/.value required,
  scholarship/.is if=record@activity@scholarship,
  sphere/.store in=\record@activity@sphere,
  sphere/.value required,
  teaching/.is if=record@entry@teaching,
  type/.store in=\record@activity@type,
  type/.value required,
  % defaults
  number,
}
\newcommand{\activity}[3][]{%
    \pgfkeys{record/activity/.cd,#1}% arguments
    {\bfseries\record@entry@getvalue[activity]{number}}
  & \pgfkeys{record/activity/.cd,#1}% arguments
    \begin{itemize}
      \item \record@entry@getvalue[activity]{date}
      \item \record@entry@getvalue[activity]{organization}
      \item \record@entry@getvalue[activity]{role}
    \end{itemize}
  & \pgfkeys{record/activity/.cd,#1}% arguments
    \begin{itemize}
      \item \record@entry@getvalue[activity]{type}
      \item \record@entry@getvalue[activity]{sphere}
    \end{itemize}
  & \pgfkeys{record/activity/.cd,#1}% arguments
    \begin{itemize}
      \item \record@entry@getvalue[activity]{frequency}
      \item \record@entry@getvalue[activity]{length}
    \end{itemize}
  & \parskip=1ex #2
  & \parskip=1ex #3
  & \pgfkeys{record/activity/.cd,#1}% arguments
    \ifrecord@entry@teaching\checkmark\fi
  & \pgfkeys{record/activity/.cd,#1}% arguments
    \ifrecord@activity@scholarship\checkmark\fi
  \\\addlinespace
}
%    \end{macrocode}
%
% \iffalse
%</package>
% \fi
%
% \Finale
\endinput

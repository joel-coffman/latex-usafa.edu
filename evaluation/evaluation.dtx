% \iffalse meta-comment
%
% Copyright (C) 2021 by Joel Coffman
%
% This file may be distributed and/or modified under the
% conditions of the LaTeX Project Public License, either version 1.2
% of this license or (at your option) any later version.
% The latest version of this license is in:
%
%   http://www.latex-project.org/lppl.txt
%
% and version 1.2 or later is part of all distributions of LaTeX
% version 1999/12/01 or later.
%
% \fi
%
% \iffalse
%<*driver>
\documentclass{ltxdoc}

\usepackage{booktabs}
\usepackage{enumitem}
\usepackage{hyperref}
\usepackage{microtype}
\usepackage{minted}

\usepackage{minted-doc}

\usepackage{evaluation}


% minted
\setminted{
  autogobble,
  breaklines,
}


\input{.version}


% macros / environments
\newcommand*{\TikZ}[0]{Ti\textit{k}Z}


\EnableCrossrefs
\CodelineIndex
\RecordChanges

\begin{document}
  \DocInput{evaluation.dtx}
\end{document}
%</driver>
% \fi
%
% \CheckSum{0}
%
% \CharacterTable
% {Upper-case   \A\B\C\D\E\F\G\H\I\J\K\L\M\N\O\P\Q\R\S\T\U\V\W\X\Y\Z
% Lower-case    \a\b\c\d\e\f\g\h\i\j\k\l\m\n\o\p\q\r\s\t\u\v\w\x\y\z
% Digits        \0\1\2\3\4\5\6\7\8\9
% Exclamation   \!     Double quote  \"     Hash (number) \#
% Dollar        \$     Percent       \%     Ampersand     \&
% Acute accent  \'     Left paren    \(     Right paren   \)
% Asterisk      \*     Plus          \+     Comma         \,
% Minus         \-     Point         \.     Solidus       \/
% Colon         \:     Semicolon     \;     Less than     \<
% Equals        \=     Greater than  \>     Question mark \?
% Commercial at \@     Left bracket  \[     Backslash     \\
% Right bracket \]     Circumflex    \^     Underscore    \_
% Grave accent  \`     Left brace    \{     Vertical bar  \|
% Right brace   \}     Tilde         \~}
%
%
% \changes{0.1.0}{2021/03/26}{Initial version}
%
% \GetFileInfo{evaluation.sty}
%
% \DoNotIndex{\#,\$,\%,\&,\@,\\,\{,\},\^,\_,\~,\ }
% \DoNotIndex{\@ne}
% \DoNotIndex{\advance,\begingroup,\catcode,\closein}
% \DoNotIndex{\closeout,\day,\def,\edef,\else,\empty,\endgroup}
% \DoNotIndex{\global,\let,\relax}
%
% \title{The \textsf{evaluation} package\thanks{This
% document corresponds to \textsf{evaluation~\fileversion-\version, dated \filedate.}}}
% \author{Joel Coffman\\\texttt{joel.coffman@usafa.edu}}
%
% \maketitle
%
% \begin{abstract}
% This package supports the creation of reports for course evaluations.
% It uses \textsf{pgfplots} to visualize raw data while imposing few constraints on the format of the report.
% \end{abstract}
%
% \section{Usage}\label{section:usage}
%
% \DescribeEnv{question}
% The |question| environment encapsulates data related to objective questions, including the question itself and responses:
%
% |\begin{question}|\marg{title} \dots |\end{question}|
%
% \noindent
% It has a single mandatory argument, \meta{title}, which is used as a header in the report.
%
% \DescribeEnv{responses}
% Within the |question| environment, the |responses| environment summarizes the possible responses:
%
% |\begin{responses}|\marg{coordinates} \dots |\end{responses}|
%
% \noindent
% The mandatory argument, \meta{coordinates}, defines the mean, and possibly error bars, of the responses.
% Multiple coordinates may be specified to compare the instructor, course, etc. against others.
%
% For example,
% \begin{VerbatimOut}[
%     gobble=2,
% ]{examples/question.tex}
% \begin{question}{Course Activities}
%   The course activities (e.g., assigned readings, lectures, discussions, labs, projects, etc.) were effective in helping me accomplish the learning goals of this course.
%
%   \begin{responses}{
%       (4.695,Course) +- (0.918,0)
%       (4.917,Department) +- (0.664,0)
%       (5.178,University) +- (1.370,0)
%   }
%     \begin{enumerate}[
%         nosep,
%     ]
%       \item Strongly Disagree
%       \item Disagree
%       \item Slightly Disagree
%       \item Slightly Agree
%       \item Agree
%       \item Strongly Agree
%     \end{enumerate}
%   \end{responses}
% \end{question}
% \end{VerbatimOut}
% \inputminted{latex}{examples/question.tex}
% which produces the following result:
%
% \noindent
% \begin{question}{ {{ label }} }
  {{ prompt }}

  \barplot[
      symbolic y coords={{ '{' }}Academy,Department,{},{{ entities.keys() | reverse | join(',') }}{{ '}' }},
  ]{
    
    ({{ values.mean }},{{ entity }}) +- ({{ values.error }},0)
    
    % aggregated statistics
    ({{ department.mean }},Department) +- ({{ department.error }},0)
    ({{ academy.mean }},Academy) +- ({{ academy.error }},0)
  }
\end{question}

%
% \noindent
% Alternatively, a table may be provided with the raw data.
% For example,
% \begin{question}{Course Activities}
%   The course activities (e.g., assigned readings, lectures, discussions, labs, projects, etc.) were effective in helping me accomplish the learning goals of this course.
%
%   \begin{responses}{
%       (4.695,Course) +- (0.918,0)
%       (4.917,Department) +- (0.664,0)
%       (5.178,University) +- (1.370,0)
%   }
%     \small
%     \begin{tabular}{llr}
%       & Response & \#\\
%       \midrule
%       1 & Strongly Disagree & 1\\
%       2 & Disagree & 3\\
%       3 & Slightly Disagree & 2\\
%       4 & Slightly Agree & 5\\
%       5 & Agree & 16\\
%       6 & Strongly Agree & 4\\
%     \end{tabular}
%   \end{responses}
% \end{question}
%
% \noindent
% Thus, the |responses| environment is fairly flexible in terms of the content it may contain.
%
% \DescribeMacro{barplot}
% A bar chart can be used to summarize feedback (as shown above).
% The |barplot| macro has a single mandatory argument, \meta{coordinates}, to specify the size of the bars (and possibly the size of error bars):
%
% |\barplot|\marg{coordinates}
%
% \noindent
% When multiple coordinates are specified, the bar chart will contain multiple bars.
% The |responses| environment (internal to the |question| environment) uses this macro to visualize the responses.
% For example, the prior visualization may be created as follows:
% \begin{VerbatimOut}[
%     gobble=2,
% ]{examples/barplot.tex}
% \barplot{
%   (4.695,Course) +- (0.918,0)
%   (4.917,Department) +- (0.664,0)
%   (5.178,University) +- (1.370,0)
% }
% \end{VerbatimOut}
% \inputminted{latex}{examples/barplot.tex}
% which produces the following graph:
%
% \noindent
% \input{examples/barplot.tex}
%
% \noindent
% Note that the width of the graph scales automatically to the available space (i.e., |\linewidth|).
%
% \StopEventually{
%   \PrintChanges
%   \PrintIndex
% }
%
% \appendix
%
% \iffalse
%<*package>
% \fi
%
% \section{Implementation}
% This section documents the implementation of the package.
%
% Require \LaTeXe.
%    \begin{macrocode}
\NeedsTeXFormat{LaTeX2e}
%    \end{macrocode}
% Identify package and version.
%    \begin{macrocode}
\ProvidesPackage{evaluation}[%
    2021/03/27 %
    v0.1.1 %
    Template for course evaluation reports%
]
%    \end{macrocode}
%
% \subsection{Packages}
% Load packages required by this one.
%
%    \begin{macrocode}
\RequirePackage{pgfkeys}
\RequirePackage{pgfplots}
%    \end{macrocode}
%
% \subsection{Macros}
% This section describes the macros in the \textsf{evaluation} package.
%
% \begin{macro}{setevaluation}
%    \begin{macrocode}
\pgfkeys{
  evaluation/.is family,
  evaluation,
    barplot/axis/.value required,
    barplot/axis/.store in=\evaluation@barplot@axis,
    barplot/axis={},
}
%    \end{macrocode}
%    \begin{macrocode}
\newcommand*{\setevaluation}[1]{%
  \pgfkeys{/evaluation,#1}%
}%
%    \end{macrocode}
% \changes{0.1.1}{2021/03/27}{%
%   Add macro for package configuration
% }
% \end{macro}
%
% \begin{macro}{barplot}
%  Define the |barplot| macro.
%    \begin{macrocode}
\newcommand*{\barplot}[2][]{%
%    \end{macrocode}
%  Create a \TikZ{} picture.
%    \begin{macrocode}
  \begin{tikzpicture}
%    \end{macrocode}
%  Define the \textsf{pgfplots} axis.
%    \begin{macrocode}
    \begin{axis}[
        axis x line=bottom,
        axis y line=left,
        axis line style={-},
        bar width=0.67\baselineskip,
        enlarge y limits=0.25,
        height=1.25in,  % TODO: Scale based on height of responses
        symbolic y coords={University,Department,Course},
        % TODO: scale exactly (see https://tex.stackexchange.com/q/36297)
        width=\linewidth - 45pt,  % 45pt is size of axis and tick labels
        xbar,
        xlabel=Mean of Responses,
        xlabel style={
          font=\footnotesize,
          yshift=1ex,
        },
        xmin=0,
        xmax=6,
        xticklabel style={
          font=\scriptsize,
        },
        ytick=data,
        yticklabel style={
          font=\footnotesize,
        },
        nodes near coords={%
          \pgfmathprintnumber[
              fixed zerofill,
              precision=3,
          ]\pgfplotspointmeta%
        },
        every node near coord/.append style={
          color=black,
          font=\scriptsize,
          opacity=0.85,
          yshift=0.75ex,
        },
        nodes near coords align={horizontal},
        \evaluation@barplot@axis,
        #1
    ]
%    \end{macrocode}
%    \begin{macrocode}
      \addplot+[
          error bars/.cd,
          error bar style={
            black,
            opacity=0.5,
          },
          x dir=both,
          x explicit,
      ] coordinates {#2};
%    \end{macrocode}
%    \begin{macrocode}
    \end{axis}
%    \end{macrocode}
%    \begin{macrocode}
  \end{tikzpicture}
%    \end{macrocode}
%    \begin{macrocode}
}
%    \end{macrocode}
% \changes{0.1.1}{2021/03/27}{%
%   Add optional argument to customize axis
% }
% \end{macro}
%
% \begin{environment}{question}
%  Define the |question| environment.
%    \begin{macrocode}
\newenvironment{question}[1]{%
%    \end{macrocode}
%  \begin{environment}{responses}
%    \begin{macrocode}
  \newenvironment{responses}[2][]{%
    \global\def\options{##1}%
    \global\def\coordinates{##2}%
    \vspace{-\medskipamount}% FORMATTING
    \vspace{-\parskip}% FORMATTING
    \noindent%
    \begin{minipage}[t]{0.45\linewidth}%
      \vspace{0pt}% hack to force vertical alignment
  }{%
    \end{minipage}%
    \begin{minipage}[t]{0.55\linewidth}%
      \vspace{0pt}% hack to force vertical alignment
      \hfill%
      \barplot[\options]{\coordinates}%
    \end{minipage}%
  }%
%    \end{macrocode}
% \changes{0.1.1}{2021/03/27}{%
%   Add optional argument to customize bar chart axis
% }
%  \end{environment}
%    \begin{macrocode}
  \paragraph{#1}
}{%
}
%    \end{macrocode}
% \end{environment}
% \iffalse
%</package>
% \fi
%
% \Finale
\endinput
